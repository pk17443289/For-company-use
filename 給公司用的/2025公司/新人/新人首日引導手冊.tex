% 使用 ctexart 文件類別,它對中文處理有很好的支援
\documentclass[a4paper, 11pt]{ctexart}

% --- 引用巨集包 ---
\usepackage{geometry}      % 用於設定頁面邊界
\usepackage{amsmath}       % 數學公式支援
\usepackage[x11names]{xcolor} % 支援更多顏色定義
\usepackage{titlesec}      % 用於自訂章節標題樣式
\usepackage{fancyhdr}      % 用於自訂頁首頁尾
\usepackage{titling}       % 用於自訂標題區塊
\usepackage{enumitem}      % 更好的列表控制
\usepackage{tcolorbox}     % 用於彩色文字框

% --- 頁面與版面設定 ---
\geometry{
  a4paper,
  top=2cm,
  bottom=2cm,
  left=2.2cm,
  right=2.2cm
}

% 調整段落間距
\setlength{\parskip}{1.2ex}

% --- 顏色定義 ---
\definecolor{guildblue}{RGB}{0, 0, 0}
\definecolor{warningred}{RGB}{200, 50, 50}
\definecolor{successgreen}{RGB}{0, 120, 70}
\definecolor{infoblue}{RGB}{41, 128, 185}

% --- 標題樣式自訂 ---
\titleformat{\section}
  {\normalfont\Large\bfseries\color{guildblue}}
  {}
  {0em}
  {}[\titlerule]

\titleformat{\subsection}
  {\normalfont\large\bfseries\color{guildblue}}
  {}
  {0em}
  {}

% --- 頁首頁尾設定 ---
\pagestyle{fancy}
\fancyhf{}
\renewcommand{\headrulewidth}{0pt}
\renewcommand{\footrulewidth}{0.4pt}
\fancyfoot[C]{\small \thepage}
\fancyfoot[R]{\small \kaishu FROM: 杰特管理團隊}

% --- 文件開始 ---
\begin{document}

% --- 抬頭資訊區 ---
\begin{center}
    \vspace{0.2cm}
    {\Huge\bfseries\kaishu 【新人引導哲學與行動手冊】}

    \vspace{0.2cm}
    {\Large\bfseries\kaishu Day 1 Onboarding: 建立實質連結的藝術}

    \vspace{0.3cm}
    \color{gray!80}
    \rule{\textwidth}{0.6pt}
    \vspace{0.3cm}
\end{center}

% --- 收件人資訊 ---
\begin{flushleft}
    \begin{tabular}{@{}ll}
        \bfseries TO: & 全體主管與新人夥伴 \\
        \bfseries FROM: & 杰特管理團隊 \\
        \bfseries VERSION: & 1.0 \\
        \bfseries PURPOSE: & 建立心理安全感的新人引導系統 \\
    \end{tabular}
\end{flushleft}
\vspace{0.3cm}

% --- 信件正文 ---

\section*{1.0 核心哲學:建立「實質連結」,而非「執行流程」}

在杰特,新人到職首日(Day 1)的\textbf{唯一核心目標},是\textbf{消除新進同仁的「孤立感」}。

我們過去的失敗,源於將「Onboarding」誤解為一份SOP的執行。新人的首要需求,不是立刻變得多有生產力,而是要確認三件事:

\begin{enumerate}[leftmargin=2em]
    \item 這個環境是\textbf{安全}的嗎?
    \item 這個環境的\textbf{規則}是清晰的嗎?
    \item 當我遇到困難時,我\textbf{敢不敢}求助,又\textbf{知道該向誰}求助?
\end{enumerate}

因此,我們所有的引導行動,都必須服務於「建立實質性的安全連結」這一最終目的。

\subsection*{1.1 必須避免的兩大「反面教材」}

我們必須摒棄過去的兩個失敗極端:

\begin{tcolorbox}[colback=red!5!white, colframe=warningred, title={\bfseries 反教材A:【冷漠的形式主義】(楷哥/鈺蓉模式)}]
\textbf{行為:} 機械性地跑完流程——介紹一圈人、辦好帳號、給他看《寶典》(內部手冊),然後說:「該知道的都在裡面了,你有問題再問。」

\textbf{診斷:} 這是一種\textbf{「防禦性交付」}。它看起來好像把所有資訊都給了,但實際上是把「學習的責任」和「提問的壓力」全部甩給了新人。這是一種「表皮」關懷,只是為了管理上的交差。
\end{tcolorbox}

\vspace{0.3cm}

\begin{tcolorbox}[colback=red!5!white, colframe=warningred, title={\bfseries 反教材B:【焦慮的過度關懷】(泓霖模式)}]
\textbf{行為:} 過度擔心新人無法存活,表現得比新人還緊張,時時刻刻都去「推」他一下。

\textbf{診斷:} 這會傳遞錯誤的信號(「你是不是很缺人?很怕我走?」),導致新人失去主動性,變成「你推一下,我才動一下」的被動等待模式。
\end{tcolorbox}

\vspace{0.3cm}

\noindent\textbf{我們的道路,是在這兩者之間,找到一條「專業、溫暖、且有邊界」的平衡道路。}

\section*{2.0 杰特引導架構:【雙角色分離系統 (Dual-Role System)】}

為了從根本上解決上述的矛盾,我們必須將新人引導的兩個衝突角色,進行制度性的拆分。

一位主管(如鈴鈺),既要考核績效(嚴厲的),又要提供安全感(溫暖的),這會導致角色衝突。因此,我們將其拆分為:

\subsection*{角色一:【直屬主管 (Supervisor)】(例如:鈴鈺)}
\begin{itemize}[leftmargin=2em]
    \item \textbf{定位:} 績效管理者 (Performance Manager)。
    \item \textbf{核心職責:} 專注於\textbf{「績效連結」}。負責傳達工作目標、教授專業SOP、考核工作產出、引導技能成長。
\end{itemize}

\subsection*{角色二:【新人夥伴 (Onboarding Buddy)】(例如:紫伶或其他資深同仁)}
\begin{itemize}[leftmargin=2em]
    \item \textbf{定位:} 安全感提供者 (Safety Provider)。
    \item \textbf{核心職責:} 專注於\textbf{「社交與文化連結」}。負責回答所有「我不好意思去問主管」的「笨問題」。
\end{itemize}

\section*{3.0 新人首日:主管的「SOP」與「行動腳本」}

作為主管(鈴鈺),妳在新人首日的任務,不是「讓他開始包貨」,而是「為他建立安全感」。

\subsection*{3.1 【硬體環境】的破冰(必須明確「賦予許可」)}

\begin{itemize}[leftmargin=2em]
    \item \textbf{廁所:} 妳必須明確地說:「在杰特,你想上廁所,\textbf{隨時都可以去}。不需要憋著,也不需要為了趕進度而犧牲自己的生理需求。」

    \item \textbf{冰箱/茶水間:} 妳必須明確地說:「冰箱是公用的,妳可以帶便當來冰。這一區的零食/飲料是公司提供的,歡迎取用。」

    \item \textbf{《寶典》:} 妳必須明確地說:「這是我們的《寶典》連結,裡面有所有關於請假、薪資的規定。妳今天不用背下來,但妳必須知道\textbf{『去哪裡查』}。」
\end{itemize}

\subsection*{3.2 【軟體環境】的建立(主動降低提問門檻)}

\begin{itemize}[leftmargin=2em]
    \item \textbf{核心問題:} 新人不敢離開「安全區」(他的工位)主動去問妳(主管)。
    \item \textbf{妳的行動:} 妳必須\textbf{主動地、有規律地「去到他身邊」}。
\end{itemize}

\begin{tcolorbox}[colback=green!5!white, colframe=successgreen, title={\bfseries 關鍵腳本(必須在第一小時內傳達)}]
「[新人名字],我需要妳非常清楚一件事:妳今天最重要的工作,不是產出,而是\textbf{『問問題』}。」

「我\textbf{允許}妳,甚至\textbf{期待}妳,可以一直問、重複問,問到妳懂為止。沒有人第一天就會。妳的提問,是在幫助我們發現流程裡不清楚的地方。」
\end{tcolorbox}

\subsection*{3.3 【夥伴制度】的啟動}

\begin{itemize}[leftmargin=2em]
    \item \textbf{妳的行動:} 正式介紹「新人夥伴」。
\end{itemize}

\begin{tcolorbox}[colback=blue!5!white, colframe=infoblue, title={\bfseries 關鍵腳本}]
「這位是[夥伴名字,例如紫伶],她是妳的『新人夥伴』。在工作上,妳主要找我;但在所有『人、環境、文化』相關的問題上,例如『中午去哪吃』、『公司網路怎麼連』、『這個笨問題我該問誰』,妳都去找她。她的任務,就是確保妳在這裡一切都順利。」
\end{tcolorbox}

\subsection*{3.4 【錯誤學習】的SOP(主管的核心技能)}

\begin{itemize}[leftmargin=2em]
    \item \textbf{情境:} 當新人犯了第一個(非毀滅性的)錯誤時。
    \item \textbf{妳的行動:} 妳的職責,不是「糾正」,而是「引導」。
\end{itemize}

\begin{tcolorbox}[colback=blue!5!white, colframe=infoblue, title={\bfseries 關鍵腳本}]
\begin{enumerate}
    \item \textbf{(停止責備):}「好的,我看到這個問題了。沒關係,這很正常,我們來一起看看。」

    \item \textbf{(分析原因):}「妳覺得,是什麼原因導致我們犯了這個錯?是SOP寫不清楚?還是標示有問題?」

    \item \textbf{(賦予任務):}「非常好,我們發現是SOP的圖片有歧義。我現在交給妳一個很重要的任務:\textbf{請妳,作為這個錯誤的發現者,去幫我把這份SOP修改得更清晰},確保下一個人不會再犯同樣的錯。完成後,交給泓霖審核歸檔。」
\end{enumerate}
\end{tcolorbox}

\vspace{0.3cm}

\noindent\textbf{文化意義:} 我們在用這個行動,向新人證明:\textbf{在杰特,犯錯是安全的,並且,犯錯是妳對組織做出貢獻的最佳途徑。}

\section*{4.0 成功的衡量:從「感覺」到「數據」}

我們如何判斷一次Onboarding是否成功?

\subsection*{最終指標(我們的哲學)}
當新人下班時,他內心篤定地相信:「當我明天遇到問題時,我\textbf{知道}該找誰,而且我\textbf{敢}去找他。」

\subsection*{量化指標(我們的工程)}
我們將在新人到職的\textbf{第3天、第1週、第30天},由主管(或HR)進行15分鐘的Check-in,並記錄以下關鍵指標:

\begin{enumerate}[leftmargin=2em]
    \item \textbf{「從1到10分,你感覺『安全地』向主管提問(關於工作)的程度有多高?」}
    \item \textbf{「從1到10分,你感覺『安全地』向夥伴提問(關於文化)的程度有多高?」}
    \item \textbf{「在過去一週,你是否曾經有過『想問問題,但最終沒敢問出口』的時刻?(是/否)」}
\end{enumerate}

\vspace{0.5cm}

\begin{center}
\rule{0.8\textwidth}{0.4pt}
\end{center}

\vspace{0.3cm}

\section*{本章總結}

\begin{tcolorbox}[colback=yellow!10!white, colframe=black, boxrule=0.5pt]
主管的職責,是成為\textbf{「環境的搭建者」}與\textbf{「安全的賦予者」}。

新人引導的成功,不在於新人第一天學會了多少,而在於我們第一天,為他注入了多少「心理安全感」。
\end{tcolorbox}

\vspace{1cm}

\begin{flushright}
    \Large\kaishu 讓每一位新人,都能感受到杰特的溫暖。
\end{flushright}

\end{document}
